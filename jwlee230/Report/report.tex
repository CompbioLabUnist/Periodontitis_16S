% !TeX spellcheck = en_US
% !TeX encoding = UTF-8
\documentclass[a4paper]{article}
\usepackage{graphics, graphicx}
\usepackage{fancyvrb, enumerate}
\usepackage{amsmath, amssymb, amscd, amsfonts}
\usepackage{geometry}
\usepackage{multirow}
\usepackage{url}
\usepackage{listings, listing}
\usepackage{color}
\usepackage{mathptmx}
\usepackage[numberedbib]{apacite}
\usepackage[style=iso]{datetime2}

\geometry
{
    top = 20mm,
    bottom = 20mm,
    left = 20mm,
    right = 20mm
}

\title{Periodontitis}
\author{
    Seunghoon Kim
    \and
    Jaewoong Lee
    \and
    Semin Lee
}
\date{\today}

\begin{document}
   	\maketitle
    \newpage

    \tableofcontents
    \listoftables
    \listoffigures
    \newpage

    \section{Introduction}
        \subsection{Microbiome}
            Microbiome is consist of microbiota, the micro-organisms which live inside and on humans \cite{microbiome1}. Microbiome is also about $10^{13}$ micro-organisms whose which collective genome \cite{microbiome2}.

            \begin{figure}[p]
                \centering
                \includegraphics[width=0.5 \linewidth]{figures/microbiome.jpg}
                \caption{Concept of a Core Human Microbiome \protect\cite{microbiome1}}
                \label{fig:microbiome-concept}
            \end{figure}

        \subsection{Ribosomal RNA}
            Ribosomal RNA (rRNA) is well-known as a key to phylogeny \cite{rRNA1}.

        \subsection{16S rRNA Gene Sequencing}

        \subsection{Periodontitis}
            Periodontitis is an inflammatory conditions which effecting periodontium, tissues  which surround and support teeth. Major components of periodontitis are clinical attachment loss and bone loss \cite{periodontitis1}. Previous study found risk factors of periodontitis such as smoking, diabetes, genetic factors and host response \cite{periodontitis2}.

    \section{Materials}
        \subsection{16S rRNA Gene Sequencing}

            \begin{itemize}
                \item 100 Healthy samples
                \item 50 Chronic Early Periodontitis Sample
                \item 50 Chronic Moderate Periodontitis Sample
                \item 50 Chronic Severe Periodontitis Sample
            \end{itemize}

    \section{Methods}
        \subsection{QIIME2 Workflow}
            QIIME2 is a capable, expandable and distributed microbiome analysis package with transparent analysis \cite{qiime1, qiime2}. A theoretic overview of QIIME2 workflow is shown as figure \ref{fig:qiime-workflow}.

            \begin{figure}[p]
                \centering
                \includegraphics[width=0.8 \linewidth]{figures/qiime.png}
                \caption{A Theoretic Overview of QIIME2 Workflow \protect\cite{qiime1, qiime2}}
                \label{fig:qiime-workflow}
            \end{figure}

            \subsubsection{Denoising techniques}
                There are two denoising techniques provided by QIIME2: DADA2 \cite{DADA1} and Deblur \cite{deblur1}. Major difference between DADA2 and Deblur is a strategy, the strategy used to divide as different species.

                \begin{figure}[p]
                    \centering
                    \includegraphics[width=0.5 \linewidth]{figures/denoising/denoising.pdf}
                    \caption{Denoising Techniques which provided by QIIME2}
                    \label{fig:denosing-workflow}
                \end{figure}

            \subsubsection{Taxonomy Classification}
                There are two taxonomy classification databases which provided by QIIME2: Greengenes (GG) \cite{greengenes1} and SILVA \cite{silva1}.

                \begin{figure}[p]
                    \centering
                    \includegraphics[width=0.5 \linewidth]{figures/taxonomy/taxonomy.pdf}
                    \caption{Taxonomy Classification which provided by QIIME2}
                    \label{fig:taxonomy-workflow}
                \end{figure}

            \subsubsection{Rarefaction}
                Rarefaction is a statistical method of estimating the number of species expected in a random sample which taken from a collection \cite{rarefaction1}. Moreover, rarefaction allows comparisons of the species richness among communities. Thus, rarefaction is a good choice for normalization \cite{rarefaction2}.

            \subsubsection{Alpha-diversity}
                Alpha-diversity is a metric which shows the richness of taxa at a single community. There are four alpha-diversity indices which provided from QIIME2:
                \begin{itemize}
                    \item Shannon's diversity index.
                    \item Observed features.
                    \item Faith's phylogenetic diversity.
                    \item Evenness index.
                \end{itemize}

            \subsubsection{Beta-diversity}
                Beta-diversity is a metric which indicates the taxonomic differentiation between multiple communities. There are four beta-diversity indices which provided from QIIME2:
                \begin{itemize}
                    \item Jaccard distance.
                    \item Bray-Curtis distance.
                    \item Unweighted UniFrac distance.
                    \item Weighted UniFrac distance.
                \end{itemize}

            \subsubsection{ANCOM}
                ANCOM (Analysis of composition of microbiomes) can be used for analyzing the composition of microbiome in multiple populations \cite{ANCOM1}. Example ANCOM volcano plot is shows as figure \ref{fig:ancom-example}.

                \begin{figure}[p]
                    \centering
                    \includegraphics[width=0.8 \linewidth]{figures/ANCOM/example.png}
                    \caption{Example ANCOM Volcano Plot which Provided by QIIME2 \protect\cite{qiime1, qiime2}}
                    \label{fig:ancom-example}
                \end{figure}

        \subsection{Python Packages}
            \subsubsection{Pandas}
                Pandas is a Python package of rich data structures and tools for analyzing with structured data sets \cite{pandas1}.

            \subsubsection{Scikit-learn}
                Scikit-learn grants state-of-the-art implementation of many machine learning algorithms, while controlling an easy-to-use interface tightly integrated the Python code \cite{sklearn1}.

            \subsubsection{Matplotlib}
                Matplotlib is a Python graphics package which used for application development, interactive scripting and publication quality image generation \cite{matplotlib2}. Matplotlib, also, is designed to create simple plots with a few commands \cite{matplotlib1}.

            \subsubsection{Seaborn}
                Seaborn is a Python data visualization package which based on matplotlib, allows a high-level interface for displaying engaging and descriptive statistical graphics \cite{seaborn1}.

    \section{Results}
        \subsection{Quality Filter}

            \begin{figure}[p]
                \centering
                $\begin{array}{cc}
                    \includegraphics[width=0.4 \linewidth]{figures/QualityFilter/Forward.png}
                    &
                    \includegraphics[width=0.4 \linewidth]{figures/QualityFilter/Reverse.png}
                    \\
                    \mbox{(a) Forward Reads} & \mbox{(b) Reverse Reads} \\
                \end{array}$
                \caption{Sequence Quality Plot}
                \label{fig:sequence-quality}
            \end{figure}

        \subsection{Rarefaction}

            \begin{figure}[p]
                \centering
                \includegraphics[width=0.6 \linewidth]{figures/Rarefaction/DADA.pdf}
                \caption{Frequency per Sample by DADA2}
                \label{fig:frequency-sample-dada2}
            \end{figure}

            \begin{figure}[p]
                \centering
                \includegraphics[width=0.6 \linewidth]{figures/Rarefaction/Deblur.pdf}
                \caption{Frequency per Sample by DADA2}
                \label{fig:frequency-sample-deblur}
            \end{figure}

        \subsection{Alpha-diversity}

            \begin{figure}[p]
                \centering
                \includegraphics[width=0.8 \linewidth]{figures/AlphaDiversity/DADA2/evenness.png}
                \caption{Evenness Index from DADA2}
                \label{fig:evenness-dada2}
            \end{figure}

            \begin{figure}[p]
                \centering
                \includegraphics[width=0.8 \linewidth]{figures/AlphaDiversity/DADA2/faith.png}
                \caption{Faith PD Index from DADA2}
                \label{fig:faith-dada2}
            \end{figure}

            \begin{figure}[p]
                \centering
                \includegraphics[width=0.8 \linewidth]{figures/AlphaDiversity/DADA2/observed.png}
                \caption{Observed Features Index from DADA2}
                \label{fig:observed-dada2}
            \end{figure}

            \begin{figure}[p]
                \centering
                \includegraphics[width=0.8 \linewidth]{figures/AlphaDiversity/DADA2/shannon.png}
                \caption{Shannon's Diversity Index from DADA2}
                \label{fig:shannon-dada2}
            \end{figure}

            \begin{figure}[p]
                \centering
                \includegraphics[width=0.8 \linewidth]{figures/AlphaDiversity/Deblur/evenness.png}
                \caption{Evenness Index from Deblur}
                \label{fig:evenness-deblur}
            \end{figure}

            \begin{figure}[p]
                \centering
                \includegraphics[width=0.8 \linewidth]{figures/AlphaDiversity/Deblur/faith.png}
                \caption{Faith PD Index from Deblur}
                \label{fig:faith-deblur}
            \end{figure}

            \begin{figure}[p]
                \centering
                \includegraphics[width=0.8 \linewidth]{figures/AlphaDiversity/Deblur/observed.png}
                \caption{Observed Features Index from Deblur}
                \label{fig:observed-deblur}
            \end{figure}

            \begin{figure}[p]
                \centering
                \includegraphics[width=0.8 \linewidth]{figures/AlphaDiversity/Deblur/shannon.png}
                \caption{Shannon's Diversity Index from Deblur}
                \label{fig:shannon-deblur}
            \end{figure}

        \subsection{Beta-diversity}

            \begin{figure}[p]
                \centering
                $\begin{array}{cc}
                    \includegraphics[width=0.4 \linewidth]{figures/BetaDiversity/DADA2/Bray/Healthy.pdf}
                    &
                    \includegraphics[width=0.4 \linewidth]{figures/BetaDiversity/DADA2/Bray/Early.pdf}
                    \\
                    \mbox{(a) Healthy} & \mbox{(b) Early} \\

                    \includegraphics[width=0.4 \linewidth]{figures/BetaDiversity/DADA2/Bray/Moderate.pdf}
                    &
                    \includegraphics[width=0.4 \linewidth]{figures/BetaDiversity/DADA2/Bray/Severe.pdf}
                    \\
                    \mbox{(c) Moderate} & \mbox{(d) Severe} \\
                \end{array}$
                \caption{Bray-Curtis Distance Index with DADA2}
                \label{fig:bray-dada2}
            \end{figure}

            \begin{figure}[p]
                \centering
                $\begin{array}{cc}
                    \includegraphics[width=0.4 \linewidth]{figures/BetaDiversity/DADA2/Jaccard/Healthy.pdf}
                    &
                    \includegraphics[width=0.4 \linewidth]{figures/BetaDiversity/DADA2/Jaccard/Early.pdf}
                    \\
                    \mbox{(a) Healthy} & \mbox{(b) Early} \\

                    \includegraphics[width=0.4 \linewidth]{figures/BetaDiversity/DADA2/Jaccard/Moderate.pdf}
                    &
                    \includegraphics[width=0.4 \linewidth]{figures/BetaDiversity/DADA2/Jaccard/Severe.pdf}
                    \\
                    \mbox{(c) Moderate} & \mbox{(d) Severe} \\
                \end{array}$
                \caption{Jaccard Distance Index with DADA2}
                \label{fig:jaccard-dada2}
            \end{figure}

            \begin{figure}[p]
                \centering
                $\begin{array}{cc}
                    \includegraphics[width=0.4 \linewidth]{figures/BetaDiversity/DADA2/UnweightedUnifrac/Healthy.pdf}
                    &
                    \includegraphics[width=0.4 \linewidth]{figures/BetaDiversity/DADA2/UnweightedUnifrac/Early.pdf}
                    \\
                    \mbox{(a) Healthy} & \mbox{(b) Early} \\

                    \includegraphics[width=0.4 \linewidth]{figures/BetaDiversity/DADA2/UnweightedUnifrac/Moderate.pdf}
                    &
                    \includegraphics[width=0.4 \linewidth]{figures/BetaDiversity/DADA2/UnweightedUnifrac/Severe.pdf}
                    \\
                    \mbox{(c) Moderate} & \mbox{(d) Severe} \\
                \end{array}$
                \caption{Unweighted Unifrac Distance Index with DADA2}
                \label{fig:unweighted-dada2}
            \end{figure}

            \begin{figure}[p]
                \centering
                $\begin{array}{cc}
                    \includegraphics[width=0.4 \linewidth]{figures/BetaDiversity/DADA2/WeightedUnifrac/Healthy.pdf}
                    &
                    \includegraphics[width=0.4 \linewidth]{figures/BetaDiversity/DADA2/WeightedUnifrac/Early.pdf}
                    \\
                    \mbox{(a) Healthy} & \mbox{(b) Early} \\

                    \includegraphics[width=0.4 \linewidth]{figures/BetaDiversity/DADA2/WeightedUnifrac/Moderate.pdf}
                    &
                    \includegraphics[width=0.4 \linewidth]{figures/BetaDiversity/DADA2/WeightedUnifrac/Severe.pdf}
                    \\
                    \mbox{(c) Moderate} & \mbox{(d) Severe} \\
                \end{array}$
                \caption{Weighted Unifrac Distance Index with DADA2}
                \label{fig:weighted-dada2}
            \end{figure}

            \begin{figure}[p]
                \centering
                $\begin{array}{cc}
                    \includegraphics[width=0.4 \linewidth]{figures/BetaDiversity/Deblur/Bray/Healthy.pdf}
                    &
                    \includegraphics[width=0.4 \linewidth]{figures/BetaDiversity/Deblur/Bray/Early.pdf}
                    \\
                    \mbox{(a) Healthy} & \mbox{(b) Early} \\

                    \includegraphics[width=0.4 \linewidth]{figures/BetaDiversity/Deblur/Bray/Moderate.pdf}
                    &
                    \includegraphics[width=0.4 \linewidth]{figures/BetaDiversity/Deblur/Bray/Severe.pdf}
                    \\
                    \mbox{(c) Moderate} & \mbox{(d) Severe} \\
                \end{array}$
                \caption{Bray-Curtis Distance Index with Deblur}
                \label{fig:bray-deblur}
            \end{figure}

            \begin{figure}[p]
                \centering
                $\begin{array}{cc}
                    \includegraphics[width=0.4 \linewidth]{figures/BetaDiversity/Deblur/Jaccard/Healthy.pdf}
                    &
                    \includegraphics[width=0.4 \linewidth]{figures/BetaDiversity/Deblur/Jaccard/Early.pdf}
                    \\
                    \mbox{(a) Healthy} & \mbox{(b) Early} \\

                    \includegraphics[width=0.4 \linewidth]{figures/BetaDiversity/Deblur/Jaccard/Moderate.pdf}
                    &
                    \includegraphics[width=0.4 \linewidth]{figures/BetaDiversity/Deblur/Jaccard/Severe.pdf}
                    \\
                    \mbox{(c) Moderate} & \mbox{(d) Severe} \\
                \end{array}$
                \caption{Jaccard Distance Index with Deblur}
                \label{fig:jaccard-deblur}
            \end{figure}

            \begin{figure}[p]
                \centering
                $\begin{array}{cc}
                    \includegraphics[width=0.4 \linewidth]{figures/BetaDiversity/Deblur/UnweightedUnifrac/Healthy.pdf}
                    &
                    \includegraphics[width=0.4 \linewidth]{figures/BetaDiversity/Deblur/UnweightedUnifrac/Early.pdf}
                    \\
                    \mbox{(a) Healthy} & \mbox{(b) Early} \\

                    \includegraphics[width=0.4 \linewidth]{figures/BetaDiversity/Deblur/UnweightedUnifrac/Moderate.pdf}
                    &
                    \includegraphics[width=0.4 \linewidth]{figures/BetaDiversity/Deblur/UnweightedUnifrac/Severe.pdf}
                    \\
                    \mbox{(c) Moderate} & \mbox{(d) Severe} \\
                \end{array}$
                \caption{Unweighted Unifrac Distance Index with Deblur}
                \label{fig:unweighted-deblur}
            \end{figure}

            \begin{figure}[p]
                \centering
                $\begin{array}{cc}
                    \includegraphics[width=0.4 \linewidth]{figures/BetaDiversity/Deblur/WeightedUnifrac/Healthy.pdf}
                    &
                    \includegraphics[width=0.4 \linewidth]{figures/BetaDiversity/Deblur/WeightedUnifrac/Early.pdf}
                    \\
                    \mbox{(a) Healthy} & \mbox{(b) Early} \\

                    \includegraphics[width=0.4 \linewidth]{figures/BetaDiversity/Deblur/WeightedUnifrac/Moderate.pdf}
                    &
                    \includegraphics[width=0.4 \linewidth]{figures/BetaDiversity/Deblur/WeightedUnifrac/Severe.pdf}
                    \\
                    \mbox{(c) Moderate} & \mbox{(d) Severe} \\
                \end{array}$
                \caption{Weighted Unifrac Distance Index with Deblur}
                \label{fig:weighted-deblur}
            \end{figure}

        \subsection{ANCOM}

            \begin{figure}[p]
                \centering
                \includegraphics[width=0.8 \linewidth]{figures/ANCOM/DADA2.gg.png}
                \caption{ANCOM Volcano Plot with DADA2 and Greengenes}
                \label{fig:volcano-dada2-gg}
            \end{figure}

            \begin{figure}[p]
                \centering
                \includegraphics[width=0.8 \linewidth]{figures/ANCOM/DADA2.silva.png}
                \caption{ANCOM Volcano Plot with DADA2 and SILVA}
                \label{fig:volcano-dada2-silva}
            \end{figure}

            \begin{figure}[p]
                \centering
                \includegraphics[width=0.8 \linewidth]{figures/ANCOM/Deblur.gg.png}
                \caption{ANCOM Volcano Plot with Deblur and Greengenes}
                \label{fig:volcano-deblur-gg}
            \end{figure}

            \begin{figure}[p]
                \centering
                \includegraphics[width=0.8 \linewidth]{figures/ANCOM/Deblur.silva.png}
                \caption{ANCOM Volcano Plot with Deblur and SILVA}
                \label{fig:volcano-deblur-silva}
            \end{figure}

    \section{Discussion}

    \bibliographystyle{apacite}
    \bibliography{reference}
\end{document}