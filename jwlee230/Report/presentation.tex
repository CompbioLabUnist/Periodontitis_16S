% !TeX spellcheck = en_US
% !TeX encoding = UTF-8
\documentclass{beamer}

\mode<presentation> { \usetheme{Madrid} }

\usepackage{graphicx, graphics}
\usepackage{apacite}
\usepackage[style=iso]{datetime2}
\usepackage{enumerate}
\DeclareGraphicsExtensions{.pdf, .png, .jpg, .gif}

\AtBeginSection[]
{
    \begin{frame}
        \vfill
        \centering
        \begin{beamercolorbox}[sep=8pt, center, shadow=true, rounded=true]{title}
            \usebeamerfont{title}
            \insertsectionhead
            \par
        \end{beamercolorbox}
        \vfill
    \end{frame}
}

\title[Periodontitis]{Periodontitis}

\author{Seunghoom Kim \and Jaewoong Lee}
\institute[UNIST]
{
    Ulsan National Institute of Science and Technology
    \medskip
    \newline
    \textit{jwlee230@unist.ac.kr}
}
\date{\today}

\begin{document}
    \begin{frame}
        \titlepage
    \end{frame}

    \begin{frame}
        \frametitle{Overview}
        \tableofcontents
    \end{frame}

    \section{Introduction}
    \begin{frame}
        \frametitle{Microbiome}

        \begin{itemize}
            \item Microbiota: the micro-organisms which live inside \& on humans \cite{microbiome1}
            \item Microbiome: about $10^{13}$ micro-organisms whose which collective genome \cite{microbiome2}
        \end{itemize}

        \begin{figure}
            \includegraphics[width=0.3 \linewidth]{figures/microbiome}
            \caption{Concept of a core human microbiome \protected \cite{microbiome1}}
            \label{fig:microbiome}
        \end{figure}
    \end{frame}

    \begin{frame}
        \frametitle{rRNA}

        \begin{itemize}
            \item Ribosomal RNA
            \item Well-known as a key to phylogeny \cite{rRNA1}
        \end{itemize}
    \end{frame}

    \begin{frame}
        \frametitle{Periodontitis (Periodontal disease)}

        \begin{itemize}
            \item CAL (Clinical Attachment Loss) \& BL (Bone Loss) \cite{periodontitis1}
            \item Risk Factors \cite{periodontitis2}
            \begin{enumerate}
                \item Smoking
                \item Diabetes mellitus
                \item Genetic factor
                \item Host response
            \end{enumerate}
        \end{itemize}
    \end{frame}

    \section{Materials}
    \begin{frame}
        \frametitle{16S rRNA Sequencing}

        \begin{itemize}
            \item 100 Healthy people
            \item 50 Chronic periodontitis -- Early
            \item 50 Chronic periodontitis -- Moderate
            \item 50 Chronic periodontitis -- Severe
        \end{itemize}
    \end{frame}

    \section{Methods}
    \begin{frame}
        \frametitle{Qiime2 Workflow}

        \begin{figure}
            \centering
            \includegraphics[width=0.7 \linewidth]{figures/qiime}
            \caption{Qiime2 Workflow \protected \cite{qiime2, qiime1}}
            \label{fig:qiime}
        \end{figure}
    \end{frame}

    \begin{frame}
        \frametitle{Taxonomy Classification}

        \begin{itemize}
            \item Greengenes (GG) \cite{greengenes1}
            \item SILVA \cite{silva1}
        \end{itemize}

        \begin{figure}
            \centering
            \includegraphics[width=0.5 \linewidth]{figures/taxonomy/taxonomy}
            \caption{Taxonomy Classification}
            \label{fig:taxonomy}
        \end{figure}
    \end{frame}

    \section{Results}

    \section{Discussion}

    \begin{frame}[allowframebreaks]
        \frametitle{References}
        \bibliographystyle{apacite}
        \bibliography{reference}
    \end{frame}
\end{document}